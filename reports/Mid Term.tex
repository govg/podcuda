\typeout{}\typeout{If latex fails to find aiaa-tc, read the README file!}
%


\documentclass[]{aiaa-tc}% insert '[draft]' option to show overfull boxes

 \title{ Computing the Proper Orthogonal Decomposition in Parallel : 
	 Study and Implementation}

	 \author{ Govind Gopakumar }

 % Data used by 'handcarry' option if invoked
 \AIAApapernumber{2014-15}
 \AIAAconference{Mid term report, BTech Project, Aerospace Engineering}
 

 % Define commands to assure consistent treatment throughout document
 \newcommand{\eqnref}[1]{(\ref{#1})}
 \newcommand{\class}[1]{\texttt{#1}}
 \newcommand{\package}[1]{\texttt{#1}}
 \newcommand{\file}[1]{\texttt{#1}}
 \newcommand{\BibTeX}{\textsc{Bib}\TeX}

\begin{document}

\maketitle

\begin{abstract}
We describe the progress in an ongoing effort to study the proper orthogonal decomposition.
The aim of the project is to study the speedups we can achieve in computing the POD in parallel,
as opposed to traditional methods. This project also includes the development of a POD library, 
implemented on GPGPUs, or General Purpose Graphic Processing Units. This is a mid term report of
the project, submitted in partial fulfillment of the requirements of the B Tech Project,
Department of Aerospace Engineering, IIT Kanpur.
\end{abstract}

\section{Introduction}

\section{Background}

\section{Motivation}

\section{Literature review}

\section{Theoretical overview}
\subsection{CUDA}
\subsection{POD}

\section{Work outline}


\begin{thebibliography}{9}% maximum number of references (for label width)
\end{thebibliography}

\end{document}

% - Release $Name:  $ -
