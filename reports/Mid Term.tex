\typeout{}\typeout{If latex fails to find aiaa-tc, read the README file!}
%


\documentclass[]{aiaa-tc_mod}% insert '[draft]' option to show overfull boxes

 \title{ Computing the Proper Orthogonal Decomposition in Parallel : 
	 Study and Implementation}

	 \author{ Govind Gopakumar }

 % Data used by 'handcarry' option if invoked
 \AIAApapernumber{2014-15}
 \AIAAconference{Mid term report, BTech Project, Aerospace Engineering}
 

 % Define commands to assure consistent treatment throughout document
 \newcommand{\eqnref}[1]{(\ref{#1})}
 \newcommand{\class}[1]{\texttt{#1}}
 \newcommand{\package}[1]{\texttt{#1}}
 \newcommand{\file}[1]{\texttt{#1}}
 \newcommand{\BibTeX}{\textsc{Bib}\TeX}

\begin{document}

\maketitle

\begin{abstract}
We describe the progress in an ongoing effort to study the proper orthogonal decomposition.
The aim of the project is to study the speedups we can achieve in computing the POD in parallel,
as opposed to traditional methods. This project also includes the development of a POD library, 
implemented on GPGPUs, or General Purpose Graphic Processing Units. This is a mid term report of
the project, submitted in partial fulfillment of the requirements of the B Tech Project,
Department of Aerospace Engineering, IIT Kanpur.
\end{abstract}

\section{Introduction}
As part of the requirements of the course AE471A, this project is being 
undertaken in two parts. Part A consists of the study of the technique known 
as the Proper Orthogonal Decomposition, and the benefits and pitfalls associated 
with implementing it on a parallel architecture. Concurrently, we develop a 
library that aims to implement the Proper Orthogonal Decomposition in parallel, 
using the CUDA environent provided by Nvidia. In this mid term report, we outline
the work that has been carried out so far, and the proposed plan for the project 
execution.


\section{Background}
The Proper Orthogonal Decomposition is used widely in different fields, for reducing the dimensionality of the data being handled. 

\section{Motivation}
There is a vast range of application for the technique known as POD. In Signal 
Processing, Data Mining, Statistical Learning, it forms a core of a set of methods
that help in dealing with large amounts of data. Specifically in computational 
fluid dynamics, we can deal with large scale descriptions of flow by proper 
decompositional modes, which significantly reduce our computational effort, while
providing us a nearly accurate solution. If there exists avenues to speed up 
the calculation of the POD, it would help boost the speed of current techniques
and help in larger adoption of this technique.


Computing the POD of a dataset is relatively fast when we use low level languages
like Fortran, and with usage of systems like MPI, we can effectively compute the
POD for large datasets in a trivial amount of time. This is possible largely due
to the presence of high performance multicore computers, which the majority of 
students do not have access to. The aim of the library developed as part of this
project is to create an easy to use and fast library for general usage, with 
possible extension into newer languages like Python. This would ensure that 
newcomers to the field can quickly utilise this library off the shelf, instead of
using Fortran code which is compartively tough to understand. 

One of the main objectives of this project is to check for possible speedups
over the existing library. Initial studies show that trivial matrix based
calculations can be optimized to provide up to a 20x speedup over traditional 
CPU based computation, and it would be one of the aims of this project to try
and obtain such massive improvements in the existing framework. 
\section{Literature review}
We give a brief overview of the current work in this field, as well as the 
relevant work which has been used to aid this project. A comprehensive overview 
shall be present in the final report.

\subsection{POD}
\subsection{POD in Aerospace Engineering}
\subsection{Parallelization and CUDA}

\section{Theoretical overview}
\subsection{CUDA}
CUDA, or Compute Unified Device Architecture is a parallel computing platform 
created by NVIDIA corp. It allows using general GPUs present in mid-high end 
computers for computaion. The advantage of using this over traditional CPU based
computation is two fold:

\begin{itemize}
\item 
	There are a large number of compute cores in a typical GPU as compared to 
	a typical CPU. A low end GPU contains upwards of few hundred cores, as 
	compared to a high end CPU which may contain less than 20 cores.
\item
	These compute cores are highly interlinked and optimized for parallel 
	execution. They are most effective when used for trivial computation 
	that can be massively parallelized.
\end{itemize}

CUDA has come to prominence of late because of the support Nvidia has released,
of developing general computing applications on their hardware. Usage of CUDA for 
scientific computing is slowly gaining favour with researchers, as it provides a 
cheaper and faster alternative to bigger larger CPU based systems.

\subsection{POD}

\section{Work outline}
We plan to carry out the project in three phases:
\begin{description}
\item[Initialization: ] This would involve a time period up to the mid semester 
	examination. The work carried out in this period would be primarily the
	initial configurations, and theoretical study. An initial literature 
	review would be carried out, to identify the relevant work, as well as 
	the state of the art. Also, the main computing system would be set up
	for development of the library. Existing code would be tracked and 
	stored for further studies.
\item[Internalization: ] This would be involve the time period from the mid 
	semester examination till October end, roughly a period of 5 weeks. 
	A concrete study of the theory would be finished by this time, and
	the program layout finalized. The most appropriate relevant research
	work would be identified as the basis for the project, and existing
	code would be analyzed thourougly, for improvements and application
	specific optimization.
\item[Realization: ] This would be carried out in the month of November, 
	culminating in the final project report, as well as release of 
	relevant code. The report would carry an overview of the results 
	obtained, the relevance of the results, all code that was developed, 
	as well as relevant documentation for the code. 
\end{description}
\begin{thebibliography}{9}% maximum number of references (for label width)
\end{thebibliography}

\end{document}

% - Release $Name:  $ -
